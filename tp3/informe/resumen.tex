\section{Resumen}
En el siguiente trabajo utilizamos el método de predicción de características conocido como Cuadrados Mínimos Lineales(CML) para analizar la predictibilidad de distintos features de nuestra base de datos, y probar las diferentes maneras con las que podemos mejorar nuestra estimación.
Se hizo un análisis exploratorio de los datos para identificar características propias de cada variable dándonos las bases para nuestro modelo y la elección de los regresores.
En las siguientes secciones presentamos los resultados e investigaciones que llevaron a concluir que un modelo que explique bien la expectativa de vida tiene que incluir variables económicas como el \textit{Income Composition of Resources} y el \textit{Percentage expenditure}, alguna variable sobre la vacunación infantil y alguna sobre la nutrición. Asimismo, la cantidad de muertes infantiles por VIH/SIDA y la cantidad de homicidios son indicadores a tener en cuenta.
También separamos al dataset en dos grandes categorías (Developed/Developing) para observar características propias de cada conjunto y encontrar regresores mas específicos para cada caso.
